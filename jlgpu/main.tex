\PassOptionsToPackage{dvipsnames}{xcolor}
\documentclass[14pt]{beamer}
\usepackage[]{xcolor}
\def\MyGreen{ForestGreen}
\def\MyRed{Red}
\def\MyOrange{BurntOrange}
\def\MyBlue{NavyBlue}
\definecolor{SuperLightGray}{rgb}{0.9,0.9,0.9}

\usepackage[]{svg}
\usepackage{amsmath}
\usepackage{mathtools}
\usepackage{emoji}
\usepackage{xfrac}
\usepackage{cancel}
\usepackage{csquotes}

\usepackage{fontspec}
\setmonofont{JetBrainsMono}[
    Path=./fonts/JetBrainsMono/,
    Scale=0.7,
    Extension = .otf,
    UprightFont=*-Regular,
    BoldFont=*-Bold,
    ItalicFont=*-Italic,
    BoldItalicFont=*-BoldItalic,
    Numbers=SlashedZero,
]


\usepackage{minted}
\usemintedstyle{bw}
\setminted{%
    autogobble=true,
    fontsize=\footnotesize,
    bgcolor=SuperLightGray,
    tabsize=2,
}


\usepackage{calc}

% This lets us \includestandalone tikz pictures in standalone documents.
\usepackage{tikz}
\usetikzlibrary{tikzmark}
\usetikzlibrary{decorations.pathreplacing,calligraphy,backgrounds}
\usetikzlibrary{calc}
\usepackage[mode=buildnew]{standalone}
\usepackage{xifthen}

\makeatother
\renewcommand{\thefootnote}{
    \ifcase\value{footnote}
        \or*
        \or**
        \or***
        \or****
    \fi}
\makeatletter

\usefonttheme{serif}
\setbeamertemplate{navigation symbols}{}%remove navigation symbols

\setbeamerfont{title}{shape=\sc}
\setbeamerfont{frametitle}{shape=\sc}

\setbeamercolor{title}{fg=black}
\setbeamercolor{frametitle}{fg=black}

\setbeamercolor{enumerate item}{fg=gray}
\setbeamerfont{enumerate item}{shape=\bf}

\newcommand{\Seq}[1]{\left\{ {#1} \right\}}
\newcommand{\Exp}[1]{e^{#1}}

\title{Julia: From multicore to GPU parallelism}
\author{\scriptsize\color{gray}
    presented by Steffen Haug}
\date{}

\begin{document}

%%
\begin{frame}
\titlepage
\end{frame}

%%
\begin{frame}
\centering
\textsc{What is Julia?}\\{\scriptsize(Briefly)}
\end{frame}

%%
\begin{frame}
    \centering
    {
        {
            \begin{minipage}[c][.9\textheight][c]{.65\textwidth}
                \raggedright\scriptsize
                \begin{enumerate}
                    \item High level programming language
                    \onslide<2- >{\item Dynamically typed}
                    \onslide<3- >{\item For HPC}
                    \onslide<4- >{\item \bf Novel compilation
                        infrastructure\\
                        {\color{gray}(Central on CPU and GPU)}}
                \end{enumerate}
                \vspace*{2em}
                \centering
                \onslide<4- >{
                    Untyped AST specialized at
                    runtime\\[1em]
                }
                \onslide<5- >{
                    {\em ``Looks like Python -- runs like
                    Fortran''}
                }
            \end{minipage}
        }
        \hspace*{\fill}
        {
            \begin{minipage}[c][.9\textheight][c]{.2\textwidth}
                \includesvg[width=\textwidth]{jllogo}
            \end{minipage}
        }
    }
\end{frame}

%%
\begin{frame}
    \centering
    \begin{minipage}[c][.9\textheight][t]{\textwidth}
        \centering
        \textsc{Example}
        \inputminted{julia}{mints/square.jl}
        \scriptsize
        \vfill
        \only<1>{Function that squares (mutates) a vector.}
        \only<2>{Not a type in sight!}
        \only<3>{When invoked with \mintinline{julia}{V},
            \mintinline{julia}{square!} is specialized to
            \mintinline{julia}{Vector{Float64}}.
        }
        \only<4>{
            \mintinline{julia}{@code_native} is a ``macro''
            that allows us to inspect code generated for a
            particular specialization.
        }
        \only<5>{
            {\bf Simple function} -- surely the assembly code is
            reasonably simple? \includegraphics{copium}
        }
    \end{minipage}
\end{frame}


%%
\begin{frame}
    \centering
    \scalebox{0.2}{
        {
            \begin{minipage}[c][5\textheight][c]{2\textwidth}
                \inputminted[tabsize=2]{text}{mints/square.asm}
            \end{minipage}
        }
    }%
    {
            \begin{minipage}[c][.95\textheight][c]{.2\textwidth}
                \centering
                \Large
                \emoji{raised-eyebrow}
                $^{\text{?}}$
            \end{minipage}
    }
\end{frame}


%%
\begin{frame}
    \centering
    \scalebox{0.4}{
        {
            \begin{minipage}[c][2.4\textheight][c]{1.5\textwidth}
                \inputminted[tabsize=2]{text}{mints/square_cleaned.asm}
            \end{minipage}
        }
    }%
    {
            \begin{minipage}[c][.95\textheight][c]{.2\textwidth}
                \centering
                \Large
                \emoji{raised-eyebrow}\\[1em]
                \scriptsize
                Remove some comments...
            \end{minipage}
    }
\end{frame}


%%
\begin{frame}
    \centering
    \scalebox{0.6}{
        {
            \begin{minipage}[c][1.1\textheight][t]{1.5\textwidth}
                \inputminted[
                    tabsize=2,
                    firstline=13,
                    lastline=36
                ]{text}{mints/square_cleaned.asm}
            \end{minipage}
        }
    }\\
    {
        \begin{minipage}[c][.2\textheight][c]{\textwidth}
            \centering\scriptsize
            Julia has {\color{\MyOrange}unrolled the loop}
            and used {\color{\MyOrange}SIMD-instructions}!
        \end{minipage}
    }
\end{frame}


%%
\begin{frame}
    \centering
    \scalebox{0.6}{
        {
            \begin{minipage}[c][1.1\textheight][t]{1.5\textwidth}
                \inputminted{c}{mints/square.c}\\
                \vspace{-3em}
                \inputminted[
                    tabsize=2,
                ]{text}{mints/square_c.asm}
            \end{minipage}
        }
    }\\
    {
        \begin{minipage}[c][.2\textheight][c]{\textwidth}
            \centering\scriptsize
            This is what Clang does.\\
            {\color{gray}({\em Slightly} better due to
            Julia-arrays being strided)}
        \end{minipage}
    }
\end{frame}


%%
\begin{frame}
    \centering
    {
    \begin{minipage}[c][.9\textheight][t]{\textwidth}
        \centering
        \textsc{Julia uses LLVM}\\[1em]
        \includegraphics[width=.5\textwidth]{LLVM_logo}\\[1em]
        \scriptsize
        Julia can do the same optimizations we
        expect in C/C++\\
        \vspace*{\fill}
        \pause
        {\em ...because it is literally the same optimizer!}
    \end{minipage}
    }
\end{frame}


%%
\begin{frame}
    \centering
    {
    \begin{minipage}[c][.8\textheight][t]{\textwidth}
        \centering
        \textsc{
            But there are some \underline{huge} caveats}\\[1em]
        
        \raggedright\scriptsize
        \begin{enumerate}
            \item \onslide<1- >{
                    There is some {\em slight} compilation overhead
                    at startup
                }
            \item \onslide<2- >{
                    Julia is {\em garbage-collected}
                    \emoji{face-vomiting}
                }
            \item \onslide<3- >{
                    We have no explicit control over memory \\
                    {\color{gray}(Allocation, copying, ...)}
                }
            \item \onslide<4- >{
                    {\em The whole thing breaks} if Julia can't
                    figure out the types
                }
        \end{enumerate}
        \centering
        \vfill
        \only<1>{
            \color{gray}However: As problem size grows, relative
            compilation time decreases.
        }
        \only<3>{
            Accidental heap operations can really hurt!
        }
        \only<4>{
            There are many ways to shoot yourself
            in the foot! \emoji{slightly-smiling-face}
        }
    \end{minipage}
    }
\end{frame}


%%
\begin{frame}
    \centering
    {
    \begin{minipage}[c][.8\textheight][t]{\textwidth}
        \centering
        {
            \scalebox{.6}{
                \begin{minipage}[c][.8\textheight][c]{.7\textwidth}
                    \centering
                    \inputminted[
                        tabsize=2,
                    ]{julia}{mints/quad_noconst.jl}\\[-.5em]
                    \vfill
                    \small
                    \onslide<2- >{
                        0.010550 seconds

                        (499.49 k allocations)
                    }
                \end{minipage}
            }
        }\hspace*{\fill}
        {
            \scalebox{.6}{
                \begin{minipage}[c][.8\textheight][c]{.7\textwidth}
                    \centering
                    \inputminted[
                        tabsize=2,
                    ]{julia}{mints/quad_const.jl}\\[-.5em]
                    \vfill
                    \small
                    \onslide<2- >{
                        0.000175 seconds

                        \phantom{(xd)}
                    }
                \end{minipage}
            }
        }\\[1em]
        
        \scriptsize
        \only<1>{Two seemingly quite similar programs...}
        \only<2>{One is $\sim\!60 \times$ faster than the other!}
        \only<3- >{And 500K allocations? \emoji{exploding-head}}
        \vfill
        \only<4>{
            \mintinline{julia}{Δx} could be reassigned
            $\implies$ Julia can't assume its type!
        }
        \only<5>{
            Leads to runtime type checks and dynamic
            dispatching...
        }
        \only<6- >{
            ...and forces intermediate calculations onto the
            heap!
        }

        \onslide<7>{
            \includegraphics{despairge}
        }
    \end{minipage}
    }
\end{frame}


%%
\begin{frame}
    \centering\scriptsize
    Julia {\em can} generate really fast code...\\[1em]
    
    \pause
    But the {\em \color{orange}lack of explicit control} 
        requires a lot of attention.\\[4em]

    \pause
    Even more so when we
    {\color{\MyBlue}parallelize} our programs.
\end{frame}


%%
\begin{frame}
    \centering
    \textsc{Multicore}\\[1em]

    \mintinline{julia}{using .Threads: @threads}
\end{frame}


%%
\begin{frame}
    \centering
    {
    \begin{minipage}[c][.8\textheight][c]{\textwidth}
        \centering
        {
            \scalebox{.6}{
                \begin{minipage}[c][.2\textheight][c]{.6\textwidth}
                    \centering
                    \inputminted[
                        tabsize=2,
                    ]{julia}{mints/nothreads.jl}
                \end{minipage}
            }
        } $\longrightarrow$
        {
            \scalebox{.6}{
                \begin{minipage}[c][.2\textheight][c]{.6\textwidth}
                    \centering
                    \inputminted[
                        tabsize=2,
                        firstline=3,
                    ]{julia}{mints/threads.jl}
                \end{minipage}
            }
        }
        \scriptsize
        Similar API to OpenMP: {\em Just annotate loops with a
        macro.}\\[2em]

        \pause
        \textsc{The normal caveats apply}
        \begin{enumerate}
            \item Spawning threads involve some overhead
            \item \pause Memory is shared (Race conditions,
                deadlocks)
            \item \pause Possible speedup severely limited
                by core count
            \item \pause Domain decomposition sometimes
                necessary to get speedup on simple problems
        \end{enumerate}
    \end{minipage}
    }
\end{frame}


%%
\begin{frame}
    \centering
    \textsc{GPU Parallelism}\\
    \includegraphics[width=.3\textwidth]{gpulogo}
\end{frame}


%%
\begin{frame}
    \centering
    \textsc{Level 1: ``Vectorized'' Array API}\\
    {\scriptsize\color{gray}(Vectorized in the sense that
    you only operate on arrays, Numpy style)}
\end{frame}

%%
\begin{frame}
    \centering
    \begin{minipage}[c][.8\textheight][t]{\textwidth}
        \centering
        \textsc{``Vectorized'' Array API}\\[1em]
        \inputminted[
            firstline=3,
            lastline=11,
        ]{julia}{mints/cuvecapi.jl}

        \vfill
        \scriptsize
        \only<1>{
            \mintinline{julia}{v},
            \mintinline{julia}{w} of type
            \mintinline{julia}{CuArray} that manages a
            device-side buffer.
        }
        \only<2>{
            Block- and gridsizes automatically determined.
        }
    \end{minipage}
\end{frame}

%%
\begin{frame}
    \centering
    \begin{minipage}[c][.8\textheight][t]{\textwidth}
        \centering
        \textsc{``Vectorized'' Array API}\\[1em]
        \inputminted[
            firstline=13,
            lastline=21,
        ]{julia}{mints/cuvecapi.jl}%

        \vfill
        \scriptsize
        \only<1>{
            Bindings to CUDA libraries like cuFFT and cuBLAS\\
            {\color{gray} (Also cuSPARSE, cuSOLVER, ...)}
        }
        \only<2>{
            GPU code generated on the fly using LLVMs
            PTX/SASS backend...\\
            {\color{gray} (Same sort of infrastructure as
            specializing Julia functions)}
        }
        \only<3>{
            Enables kernel fusion and global optimization (!)
        }
        \only<4>{
            Like with CPU code, you can inspect the
            generated GPU code.\\
            {\color{gray} (SASS/PTX code is too long/ugly to put on a slide)}
        }
    \end{minipage}
\end{frame}


%%
\begin{frame}
    \centering
    \textsc{Level 2: Julia on the GPU}
\end{frame}

%%
\begin{frame}
    \centering
    \begin{minipage}[c][.8\textheight][t]{\textwidth}
        \centering
        \textsc{Julia on the GPU}\\[1em]
        \inputminted[
            firstline=5,
        ]{julia}{mints/jlgpu.jl}
        \vfill\scriptsize
        \only<1>{
            \mintinline{julia}{cusquare!}
            is defined {\em as a regular Julia function}!\\
            {\color{gray}(It won't run if you call it;
            block- and thread-ID is not defined on the CPU)}
        }
        \only<2>{
            The \mintinline{julia}{@cuda} macro
            transforms typed {\color{\MyBlue}Julia IR}
            into {\color{\MyBlue}GPU code}!\\
            \emoji{exploding-head}
        }
        \only<3>{
            Basically, almost {\em any} sensible Julia can just work
            on the GPU\\
        }
        \only<4>{
            {However, there is no escape hatch:}
            {\em The code must
            type check, and can't allocate}
        }
    \end{minipage}
\end{frame}


%%
\begin{frame}
    \centering
    \textsc{Level 3: Multi-GPU systems}\\
    \scriptsize Future Outlook
\end{frame}


%%
\begin{frame}
    \centering
    \begin{minipage}[c][.8\textheight][t]{\textwidth}
        \centering
        \textsc{Level 3: Multi-GPU systems}\\[1em]
        \scriptsize
        \mintinline{text}{CUDA.jl} composes with
        \mintinline{text}{DistributedArrays.jl}\\[1em]

        {\raggedright
        \mintinline{text}{DistributedArrays.jl}:
        Domain decomposition and distribution algorithms \\[1em]
        \mintinline{text}{CUDA.jl}: Manages device-side
        buffers and host-device communication\\[3em]
        }

        {\color{\MyBlue}Both understands Julias abstract concept of\\ an
        ``Array'', so they work together.}


        \vfill
        {\color{gray}{\em Rapid Software Prototyping for
        Heterogeneous and Distributed Platforms}\\ Besard
        et. al.}
    \end{minipage}
\end{frame}


%%
\begin{frame}
    \centering
    \begin{minipage}[c][.8\textheight][t]{\textwidth}
        \centering
        \textsc{in summary}\\[1em]

        \scriptsize
        Julia makes bold claims%
        \only<2>{: {\em ``Looks like Python -- runs like}
        C/C++/Fortran''}%
        \only<3- >{: {\em ``Only dynamic language to achieve
        petascale''}}

        \vspace{2em}
        \only<4>{It is a truth with a \underline{big
        asterisk}.}
        \only<5->{
            If there is {\em one thing} that is true, it is
            that you can't escape\\[1em]
            \small \textsc{Location, location, location}
        }

        \scriptsize
        \vspace{2em}
        \only<6>{
            Julia has {\em real} potential, but it is
            {\em not a free lunch!}\\
            {\color{gray}(Especially on the GPU)}
        }%
        \only<7>{
            I spend a lot of time ``fighting the
            abstraction''
        }%
        \only<8->{
            Allocation, garbage-collection, copies,
            CUDA device$\leftrightarrow$host transfers,
            etc.\\
            are all {\em implicit!}\\[1em]
        }%
        \only<9>{
            As Julia takes away your control, you have to be
            careful.
        }%
        \only<10>{
            {\color{\MyBlue} The abstraction leaks!}
        }%
    \end{minipage}
\end{frame}


%%
\begin{frame}
    \centering
    \begin{minipage}[c][.8\textheight][t]{\textwidth}
        \centering
        \textsc{in summary}\\[1em]

        \scriptsize
        \only<1-3>{
            {\em What niche does Julia even fill?}\\[2em]
        }%
        \only<2>{
            If you {\em do understand} memory, you don't want it.
        }%
        \only<3>{
            If you {\em don't understand} memory, you
            basically can't use it!
        }%
    \end{minipage}
\end{frame}

\end{document}
